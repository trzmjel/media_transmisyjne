\subsection{Wyjaśnienie}
\paragraph{}
Obliczenia były wykonane w celu upewnienia się, że wszystkie elementy wybrane w naszym projekcie będą skutkowały łącznie funkcjonalną siecią. Pozostałe obliczenia w tym zadaniu zostały wykonane w celu sprawdzenia, jakie alternatywny mogłyby ewentualnie zostać wprowadzone do projektu z racji na np. konieczność korzystania ze starszej infrastruktury, posiadania na stanie starszego sprzętu itp.

\begin{table}
	\centering
	\caption[]{Założenia obliczeniowe}
	\begin{adjustbox}{width=1\textwidth}
	\begin{tblr}{
			width = \linewidth,
			colspec = {Q[60]Q[127]Q[65]Q[104]Q[71]Q[102]Q[81]Q[127]Q[100]Q[88]},
  			cells = {c},
			row{even} = {Silver},
			hlines,
			vlines
		}
		\textbf{-}                  & \textbf{Budżet mocy optycznej systemu transmisyjnego (L)} & \textbf{Liczba złączy (a)} & \textbf{Średnia wartość tłumienia złącza (C)} & \textbf{Liczba spawów (b)} & \textbf{Średnia wartość tłumienia spawów (S)} & \textbf{Tłumienie sprzęgacza (BD)} & \textbf{Tłumienność jednostkowa włókna światłowodowego (alfa)} & \textbf{Maksymalna długość toru optycznego (l)} & \textbf{Margines projektowy (m)} \\
		\textbf{Test zasięg}        & 28                                                        & 4                          & 0,2                                           & 6                          & 0,1                                           & 18,6                               & 0,3                                                            & 0,9                                             & 3                                \\
		\textbf{Test tłumienie}     & 28                                                        & 4                          & 0,5                                           & 6                          & 0,1                                           & 18,6                               & 0,3                                                            & 3                                               & 3                                \\
		\textbf{Obliczenie pkt 2/4} & 28                                                        & 2                          & 0,3                                           & 3                          & 0,1                                           & 21                                 & 0,3                                                            & 0,9                                             & 3                                \\
		\textbf{APON}               & 28                                                        & 2                          & 0,3                                           & 3                          & 0,1                                           & 21                                 & 0,3                                                            & 0,9                                             & 3                                \\
		\textbf{EPON}               & 24                                                        & 2                          & 0,3                                           & 3                          & 0,1                                           & 21                                 & 0,3                                                            & 0,9                                             & 3                                \\
		\textbf{XG-PON}             & 31                                                        & 2                          & 0,3                                           & 3                          & 0,1                                           & 21                                 & 0,3                                                            & 0,9                                             & 3                                \\
		\textbf{I okno}             & 28                                                        & 2                          & 0,3                                           & 3                          & 0,1                                           & 21                                 & 3                                                              & 0,9                                             & 3                                \\
		\textbf{III okno}           & 28                                                        & 2                          & 0,3                                           & 3                          & 0,1                                           & 21                                 & 0,2                                                            & 0,9                                             & 3                                \\
		\textbf{IV okno}            & 28                                                        & 2                          & 0,3                                           & 3                          & 0,1                                           & 21                                 & 0,1                                                            & 0,9                                             & 3                                \\
		\textbf{1:64}               & 28                                                        & 2                          & 0,3                                           & 3                          & 0,1                                           & 21,3                               & 0,3                                                            & 0,9                                             & 3                                \\
		\textbf{1:32}               & 28                                                        & 2                          & 0,3                                           & 3                          & 0,1                                           & 17,3                               & 0,3                                                            & 0,9                                             & 3                                \\
		\textbf{1:16}               & 28                                                        & 2                          & 0,3                                           & 3                          & 0,1                                           & 13,6                               & 0,3                                                            & 0,9                                             & 3                                \\
		\textbf{1:2, 1:2, 1:16}     & 28                                                        & 2                          & 0,3                                           & 3                          & 0,1                                           & 21,6                               & 0,3                                                            & 0,9                                             & 3                                
	\end{tblr}
\end{adjustbox}
\end{table}

\begin{table}
	\centering
	\caption[]{Analiza}
	\begin{tblr}{
			width = \linewidth,
			colspec = {Q[446]Q[177]Q[262]},
			row{even} = {Silver},
			hlines,
			vlines,
		}
		-                           & \textbf{Zasięg [km]} & \textbf{Tłumienie [dB]} \\
		\textbf{Test zasięg}        & 16,67           & 20,27              \\
		\textbf{Test tłumienie}     & 12,67           & 22,1               \\
		\textbf{Obliczenie pkt 2/4} & 10,33           & 22,17              \\
		\textbf{APON}               & 10,33           & 22,17              \\
		\textbf{EPON}               & -3,00           & 22,17              \\
		\textbf{XG-PON}             & 20,33           & 22,17              \\
		\textbf{I okno}             & 1,03            & 24,6               \\
		\textbf{III okno}           & 15,50           & 22,08              \\
		\textbf{IV okno}            & 31,00           & 21,99              \\
		\textbf{1:64}               & 9,33            & 22,47              \\
		\textbf{1:32}               & 22,67           & 18,47              \\
		\textbf{1:16}               & 35,00           & 14,77              \\
		\textbf{1:2, 1:2, 1:16}     & 8,33            & 22,77              
	\end{tblr}
\end{table}